%
%	Einfuehrung
%

\pagebreak
\section{Introduction}

\onehalfspacing

\subsection{Online Marketing and Advertising}

The web is watching you. Almost everywhere you go online, advertisers follow, gathering data and using it to show you ads for products you might be persuaded to buy. Since the mid-1990s, that surveillance infrastructure has relied on tracking technology called third-party cookies.

From news articles to web comics to cat videos, the internet has more media than you could ask for, much of it available for you to access without paying a dime.

But nothing in life is free, and the internet is no exception. In reality, there’s a deal happening every time you consume a piece of free content online. Web publishers are giving you their content; advertisers agree to fund that content by paying for ads. And you agree to give mountains of personal data to hundreds of companies in the digital ad industry, whether you realize it or not.

That’s the bargain that has funded free web publishing since the mid-1990s, and for the past quarter century it’s been powered by a key piece of technology known as third-party cookies.

These are tiny but crucial identifiers that track internet users’ every move across the web. They help advertisers target ads and measure the effectiveness of their marketing campaigns. They’ve become one of the central technologies underpinning the business model of publishing on the web.

And they’re about to die.

\subsection{Gender-neutral Pronouns}

Our society is becoming more open, inclusive, and gender-fluid, and now I think it's time to think about using gender-neutral pronouns in scientific texts, too. Two well-known researchers, Abigail C. Saguy and Juliet A. Williams, both from UCLA, propose to use singular they/them instead: "The universal singular they is inclusive of people who identify as male, female or nonbinary."\footnote{\textit{Saguy, A. (2020)}: Why We Should All Use They/Them Pronouns. \cite{pronouns}} The aim is to support an inclusive approach in science through gender-neutral language. 

In this paper, I'll attempt to follow this suggestion and invite all my readers to do the same for future articles. Thank you!

If you're not sure about the definitions of gender and sex and how to use them, have a look at the definitions\footnote{See \textit{APA (2021)}: Definitions Related to Sexual Orientation. \cite{apaDefinitions}} by the American Psychological Association.

