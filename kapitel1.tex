%
%	Einfuehrung
%

\pagebreak
\section{Introduction}

\onehalfspacing

\subsection{Online Marketing and Advertising}

Advertising is the fuel that powers the web. As much as we hate to admit, it's the advertising revenues that fund most of the services available to us on the Internet, from Search to News and Weather. And cat pictures.

The main goal for online marketing and advertising is to increase sales by placing ads that are meaningful to the viewer. To do this, advertising agencies and publishers collect as much data on the viewer as they can, to make sure the advertising is as relevant as possible.

A key technical element in this is the so-called cookie, a piece of information that a website can store on the users' computer, or more precisely, inside the user's browser. These cookies can be evaluated by the site placing them, or by third-parties. With many sites placing cookies and many agencies collecting the data, a complete profile of the user behind the browser emerges and enables the advertising agencies to target their potential customers very precisely.

This type of targeting could not only be used for advertising, but also to exert political influence and affect elections. For the purpose of this paper, we'll focus on the advertising aspect.

The vast amount of data available in user profiles is of concern for privacy experts and several geographies have begun to enact privacy laws, to curb the data collection.

A very contentious point are third-party cookies, which we will analyze in more detail throughout this paper. We will focus on two key aspects, the technical implementation and possible alternatives as well as the legal framework in the European Union for collection such data.

Being able to show meaningful advertising is key for the commercial viability of the advertising business, and thus for the world wide web as a whole. The current changes is tracking cookies will have a big effect on the industry and we want to use this paper to gain some insights.

The overall situation is still very much in flux and there is no clear consensus on a way forward; very recently Google has announced to extend the life of third-party cookies until the end of 2023.\footnote{See \textit{Goel, V. (2021)}: An updated timeline for Privacy Sandbox milestones. \cite{sandboxDelay}}

In this paper I will focus on the current state at the time of writing and rely on expert opinions from the field of online marketing.

\subsection{Gender-neutral Pronouns}

Our society is becoming more open, inclusive, and gender-fluid, and now I think it's time to think about using gender-neutral pronouns in scientific texts, too. Two well-known researchers, Abigail C. Saguy and Juliet A. Williams, both from UCLA, propose to use singular they/them instead: "The universal singular they is inclusive of people who identify as male, female or nonbinary."\footnote{\textit{Saguy, A. (2020)}: Why We Should All Use They/Them Pronouns. \cite{pronouns}} The aim is to support an inclusive approach in science through gender-neutral language. 

In this paper, I'll attempt to follow this suggestion and invite all my readers to do the same for future articles. Thank you!

If you're not sure about the definitions of gender and sex and how to use them, have a look at the definitions\footnote{See \textit{APA (2021)}: Definitions Related to Sexual Orientation. \cite{apaDefinitions}} by the American Psychological Association.

Also, to be mindful in our writing, requires careful evaluation of the terminology we use in our text and a certain level of restraint. For this paper, I use a handy reference to ableist terms\footnote{See \textit{Brown, L.X.Z. (2021)}: Ableism/Language. \cite{ableismLanguage}} that I want to avoid.

\subsection{Tools}

I wrote the LaTeX \href{https://github.com/chfrank-cgn/Bachelor-Arbeit}{source code} for this thesis with \href{https://www.overleaf.com/}{Overleaf}, I checked for grammar, style, and plagiarism with \href{https://app.grammarly.com/}{Grammarly}, and I had \href{https://www.youtube.com/watch?v=5qap5aO4i9A}{LofiGirl} for the soundtrack of my writing.

\subsection{Acknowledgments}

I am very grateful to \href{https://www.linkedin.com/in/yvonne-romes/}{Yvonne Romes}, lecturer at FOM, for her support on this thesis; I am extremely thankful and indebted to her for sharing expertise, and sincere and valuable guidance and encouragement extended to me.

I am also very grateful to the members of my expert panel, without their generous and patient support this paper would not have been possible.
