%
%	Theorieteil
%

\pagebreak
\section{Research Methodology}

\onehalfspacing

\subsection{Developing Situation}

The current situation in regards to targeting and tracking cookies as enabling technology is under heavy development and undergoing changes almost on a daily basis; for this paper we will look at developments up until the end of July 2021 and hope that by the time you read this it will still be somewhat relevant.

Literature is still sparse, and thorough quantitative analysis has yet to be done. To deliver the most value and reach an answer to our research question on which strategies we can develop to place meaningful advertisement in the future, in a DS-GVO compliant way and without using cookies, we're choosing a qualitative research method.

We want to place an emphasis on technical and legal aspects and hence will focus our research on these aspects and aim to reach a conclusion at the end of our analysis.

\subsection{Content Analysis}

The most fitting method for the research at hand is Qualitative Content Analysis as pioneered by Philipp Mayring, who developed a framework to perform structured qualitative analysis of text through content analysis.\footnote{See \textit{Mayring, P. (2020)}: Qualitative Content Analysis \cite{qualiContent}} During their tenure at the university of Klagenfurt, Phillip Mayring defined a stringent procedural approach on how to analyze text and deduct research information from it, using inductive development of categories and subsequently the deductive application of these.

\subsection{Expert Interview}

Rather than analyzing the myriads of available opinion texts on the web, we decided to use a panel of experts, to gather the most up-to-date expert knowledge and provide guidance and value. Our approach of conducting expert interviews is based on Robert Kaiser's book on the same topic.\footnote{See \textit{Kaiser, R. (2014)}: Qualitative Experteninterviews \cite{expertInterviews}}

Especially since there is not yet an established field of expertise and literature on this topic available, using an expert panel seems to be warranted and the most obvious choice for our research. It will give us the opportunity to differentiate between all available options and gain clarity and insight on the topic.

\subsection{Summative Content Analysis} 

The most prevalent method of performing a qualitative content analysis on open-ended interviews is to build categories, as outlined by Philipp Mayring.\footnote{See \textit{Mayring, P. (2020)}: Qualitative Content Analysis \cite{qualiContent}}

In our case, given the rather broad subject and the small panel of experts, we will, however, be using a different technique instead: Content analysis through structure formation techniques.\footnote{See \textit{Kindermann, K. (2020)}: Summative Content Analysis \cite{summaContent}} We will build the answers to our research question by following and summarizing the answers of our experts. This approach proposed by Katharina Kindermann of treating every interview as a separate set of data is especially helpful in a situation like ours, where the expert panelists come from very different backgrounds and offer a quite diverse view of the situation at hand.

\subsection{Critique of Methodology}

Using expert interviews is not without issues, as Robert Kaiser points out, \footnote{See \textit{Kaiser, R. (2014)}: Qualitative Experteninterviews, p. 125 \cite{expertInterviews}} and we need to look at a couple of possible problems before starting the analysis.

One of the possible problems Robert Kaiser points out is the missing justification for conducting expert interviews, as opposed to a thorough analysis of literature.\footnote{See \textit{Kaiser, R. (2014)}: Qualitative Experteninterviews, pp. 126-128 \cite{expertInterviews}} As were are trying to develop strategies on a very new and still volatile situation, the use of an expert panel seems justified. Even though there's a lot of articles on the subject available on the web, no clear consensus has yet emerged and expert analysis is spread far and wide.

Another problem could possibly arise from the selection of the experts on the panel, according to Robert Kaiser.\footnote{See \textit{Kaiser, R. (2014)}: Qualitative Experteninterviews, pp. 132-136 \cite{expertInterviews}} By using a rather diverse panel and focus on practical experience in the field, we hope to avoid this. Our three panelists cover all relevant technical and legal aspects, but do not belong to the same opinion bubble, so that we can be relatively sure that we're avoiding too strong a bias here. We're also conducting the interviews one by one, to avoid that our experts influence each other.

The final problem that we want to address, is a possible lack of theory in the analysis and a too high an emphasis on the interview itself.\footnote{See \textit{Kaiser, R. (2014)}: Qualitative Experteninterviews, pp. 144-146 \cite{expertInterviews}} Focusing on possible strategies to deal with the demise of the tracking cookie has indeed potential to stray from the wider field of knowledge available. We have laid a solid foundation in the previous chapter, though, and the outcome of the analysis will be within the parameters set by the research question and relevant for it.

\subsection{Expert Panel}

To get a balanced view and a broad range of expertise, we selected three well-knows experts in their field, from Cologne and the surrounding areas. All three are highly regarded experts in their field and fully qualified to offer insight and guidance on the subject:

\begin{itemize}
 \item \href{https://www.linkedin.com/in/renate-schmid-535233113/}{Renate Schmid}
 \item \href{https://www.linkedin.com/in/roman-pusep-36b33374/}{Roman Pusep}
 \item \href{https://www.linkedin.com/in/eicker/}{Gerrit Eicker}
\end{itemize}

Renate is a lawyer at \href{https://www.wbs-law.de/}{Wilde Beuger Solmecke}, a renowned and leading law firm in the field of media and copyright law, located in Cologne. With \href{https://www.youtube.com/user/KanzleiWBS}{Kanzlei WBS} they operate the biggest German-language law channel on YouTube and are widely recognized as experts in their field. Throughout the last couple of years, WBS has actively participated in the discussions about copyright law and Article 17 (the article formerly known as 13)\footnote{See \textit{Solmecke, C. (2021)}: Die Uploadfilter sind da \cite{article17}} and are known as thought-leaders in the field of e-privacy.

Roman is a partner in the law firm \href{https://www.werner-ri.de/}{Werner Rechtsanwälte Informatiker}. Werner RI is a law firm providing legal advice on commercial law, with a strong focus on IT, data protection and internet law; they are widely known as thought-leaders on law in the field of digitization and frequently offer seminars and workshops on the intersection of IT and law.

Gerrit is the owner of the agency \href{https://eicker.digital/}{eicker.digital - Wir sprechen Online}. The agency is focused on digital communication and is known as a thought-leader in the field of digitization and conversion. With \href{https://www.youtube.com/eickertv}{eicker.TV} they run a well-known German-language channel for up-to-date information on digital marketing and communication and regularly publish on the major marketing networks.

Between the tree of them we are fortunate to have covered a broad range of expertise on the fields that we need to look into for the future of targeted advertising, from data protection laws and civic society all the way down to digital communication. The diverse panel will guarantee a well-rounded outcome.

\subsection{Interview Questionnaire}

To conduct an expert interview, we need to create a questionnaire. With the technology being in flux and the legal framework still developing, we want to focus our panel questions on the the one aspect that is not likely to change, the three different type of data that we have outlined above.

Regardless of future development, we can safely assume that this classification will hold for a couple of years and will be a good basis for further analysis. The three types of data are independent of any technology; they are also treated differently in current legislation.

For the interview question, we'll stick to using direct questions only, as we want to gather as much information from our panelists as possible, without leading them towards a certain outcome or opinion.\footnote{See \textit{Kaiser, R. (2014)}: Qualitative Experteninterviews, p. 68 \cite{expertInterviews}} 

Firstly, we start with asking about first party data. We want to gather information on how our experts view the field of first party data in the current environment and in the near future. 

\begin{itemize}
 \item In the current environment, from your point of view, what's the primary legal basis for collecting first-party data?
 \item In the near future (one to three years), how do you see the development of the legal situation for collecting first-party data?
\end{itemize}
 
From the answers we expect additional information on the current legal frameworks, data privacy aspects, and a commercial view from a marketing viewpoint. Also, we expect an outlook on the possible developments in the next couple of years.
 
Secondly, we ask the same question again, this time for second party data:

\begin{itemize} 
 \item In the current environment, from your point of view, what's the primary legal basis for obtaining second-party data?
 \item In the near future (one to three years), how do you see the development of the legal situation for obtaining second-party data?
\end{itemize}

Here, we're expecting additional information especially on the possible roles of trustees and data processors, in addition on more general views on second party data itself and on how it could possibly play a bigger role in online marketing in the future.
 
Thirdly, we ask the final set of questions on third party data:

\begin{itemize} 
 \item In the current environment, from your point of view, what's the primary legal basis for obtaining third-party data?
 \item In the near future (one to three years), how do you see the development of the legal situation for obtaining third-party data?
\end{itemize} 
 
The use of third party data is the most contentious usage, here we are looking for more insights especially into the privacy aspects, together with more general views on the legality of the current approaches and possible options for the future.

To finish, we'll ask the experts to speculate about the future of targeting in online marketing:
 
\begin{itemize} 
 \item Taking a wild guess, from a legal point of view, where do you think targeting on the web will be in ten years?
\end{itemize}

We do not have any firm expectations here, we just want to learn what our experts think about the future of online marketing and where they see the development going.

\subsection{Interview Execution}

The interview with Renate Schmid was conducted in English through Email,

The interviews with Roman Pusep and Gerrit Eicker were conducted each over a 90 minutes video call in German. The call was made and recorded with \href{https://zoom.us/}{Zoom} and then manually transcribed using \href{https://f4x.audiotranskription.de/}{f4x Spracherkennung} as a starting point. 

Where necessary for the analysis, translation from German to English was performed with support from \href{https://www.deepl.com/en/translator}{DeepL}.

In the next and final chapter, we will now analyze the outcomes of our expert interviews.
