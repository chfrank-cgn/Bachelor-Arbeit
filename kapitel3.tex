%
%	Theorieteil
%

\pagebreak
\section{Research Methodology}

\onehalfspacing

\subsection{The Need for Experts}

The current situation in regards to targeting and tracking cookies as one enabling technology is under heavy development and undergoing changes almost on a daily basis; for this paper we will look at developments up until the end of July 2021 and hope that by the time you read this paper, the results will still be somewhat relevant.

Available Literature is still sparse at the time of writing, and thorough quantitative analysis has yet to be performed. Still, we want to analyze the situation and come up with strategies for online marketing in a (tracking) cookie-less world.

Our goal is to to reach a conclusion at the end of our analysis and present a couple of possible strategies for advertising placement in the future. We will place an emphasis on the technical and legal aspects and hence will focus our research on these; our aim is to come up with strategies that are both DS-GVO/GDPR compliant and respecting the user's privacy.

These strategies will be the outcome of our research and the answer to the research questions in the final chapter of this paper.

To perform our analysis and deliver the most value, we're choosing Content Analysis for our research, a qualitative research method.

\subsection{Content Analysis}

The most fitting method for the research at hand is Qualitative Content Analysis as pioneered by Philipp Mayring. In 1983, Philipp Mayring developed a framework to perform structured qualitative analysis of text through content analysis. During their tenure at the university of Klagenfurt, Phillip Mayring defined a stringent procedural approach on how to analyze text and deduct research information from it, using inductive development of categories and subsequently the deductive application of these.\footnote{See \textit{Mayring, P. (2020)}: Qualitative Content Analysis \cite{qualiContent}}

The method is well respected in the field and accompanied by a lot of background literature on how to perform such an analysis and arrive at meaningful results.\footnote{See \textit{Marx, J. (2019)}: Qualitative Inhaltsanalyse nach Mayring \cite{qualiInhalt}}

Mayring develop this method of qualitative content analysis as an addition to a pure quantitative content analysis, with the aim to establish a framework to perform a systematic analysis of text. To do this, Mayring proposed to establish a set of categories in which the text is then analyzed and summarized. The interpretation at the end will then lead to the desired result of the analysis.\footnote{See \textit{Halbmayer, E. (2010)}: Qualitative Inhaltsanalyse \cite{qualiVienna}}

The analysis is not only tied to the content of analyzed text, but can also take other factors into account, such as structure or context.\footnote{See \textit{Mayring, P. (2020)}: Qualitative Content Analysis \cite{qualiContent}} For the purpose of this paper, we'll focus only on the content and not factor in structure or context.

\subsection{Expert Interview}

Rather than analyzing only the myriads of available opinion texts on the web, we decided to use a panel of experts for further analysis. We aim to gather the most up-to-date expert knowledge. to provide guidance and value. Our approach of conducting expert interviews is mainly based on Robert Kaiser's book on the same topic.\footnote{See \textit{Kaiser, R. (2014)}: Qualitative Experteninterviews \cite{expertInterviews}}

The first question we need to look at, is what constitutes an expert. Uwe Flick advocates a rather open definition and suggests to look at the person's role or function within the field of where the expertise is required, rather than at formal or academic qualification.\footnote{See \textit{Flick, U. (2009)}: Qualitative Sozialforschung \cite{expertDefinition}}

Especially since there is not yet any academic literature available around the demise of the tracking cookie and its effect on online marketing, using a panel of experts seems to be warranted and the most obvious choice to conduct our research. It will give us the opportunity to differentiate between all available options and gain clarity and insight on the topic. It will also allow us to capture the most up-to-date information and achieve valuable and actionable results and outcomes.

From the available people in the field, we were fortunately able to secure a couple of well-known experts, which we will introduce below. The field of expertise we were looking at were the legal and technical aspects of data driven online marketing, matching the focus of the underlying research question. All of our experts have a strong professional relation to the field of digital online marketing.

What's an expert interview? In its essence, an expert interview is a conversation with one or more persons.\footnote{See \textit{Bayer, M. (2021)}: Expert Interview for College/University Students \cite{whatIsInterview}} For our case, we will split the interviews into different one-on-one session, rather than using a group setting, to cater for the busy schedules of the individuals involved. Further below, after the introduction of our expert panel, we will also outline the interview questions and the rationale behind each of them.

\subsection{Summarizing Content Analysis} 

The most prevalent method of performing a qualitative content analysis on open-ended interviews is to build categories and then abstract the results, as outlined by Philipp Mayring.\footnote{See \textit{Mayring, P. (2020)}: Qualitative Content Analysis \cite{qualiContent}}

In our case, given the rather broad subject and the small panel of experts, we will, however, be using a different technique instead: Content analysis through structure formation techniques\footnote{See \textit{Kindermann, K. (2020)}: Summative Content Analysis \cite{summaContent}} and perform a summarizing content analysis.

Goal of a summarizing content analysis is to reduce the material, in our case the answers to our interview questions, in a way that the original meaning is kept but the extent is reduced down to to a manageable level.\footnote{See \textit{Halbmayer, E. (2010)}: Qualitative Inhaltsanalyse \cite{summaryVienna}}

For each of our interviews, we will first paraphrase the answers by question, to distill the most important information from our experts, and then generalize a summative answer for each of the questions. This approach proposed by Katharina Kindermann of treating every interview as a separate set of data is especially helpful in a situation like ours, where the expert panelists come from very different backgrounds and can offer a quite diverse view of the situation at hand.\footnote{See \textit{Kindermann, K. (2020)}: Summative Content Analysis \cite{summaContent}}

From the generalized answers to the research questions we will then build a summative conclusion to our expert interviews, which in Phillip Mayring's process definition constitutes the final integration step.\footnote{See \textit{Halbmayer, E. (2010)}: Qualitative Inhaltsanalyse \cite{summaryVienna}} We will then be using a dialectic approach to reach the outcome for our research question.\footnote{See \textit{Adorno, T.W. (2019)}: Einführung in die Dialektik \cite{introDialectic}}

\subsection{Critique of Methodology}

Using expert interviews is not without issues, as Robert Kaiser points out, \footnote{See \textit{Kaiser, R. (2014)}: Qualitative Experteninterviews, p. 125 \cite{expertInterviews}} and we need to address a couple of possible problems before starting the analysis.

One of the possible problems Robert Kaiser points out is the missing justification for conducting expert interviews, as opposed to a thorough analysis of literature.\footnote{See \textit{Kaiser, R. (2014)}: Qualitative Experteninterviews, pp. 126-128 \cite{expertInterviews}} As were are trying to develop strategies on a very new and still pretty volatile situation, the use of an expert panel seems to be justified. Even though there are a lot of articles on the subject available on the web, no clear consensus has yet emerged and expert opinions are spread far and wide. Using well-known experts in the field to perform a thorough research and reach a consensus was the most logical approach.

Another issue could be to determine whether we are looking for the right knowledge or expertise in our experts.\footnote{See \textit{Kaiser, R. (2014)}: Qualitative Experteninterviews, pp. 128-132 \cite{expertInterviews}} All our experts are practitioners in the legal or marketing profession and we are looking for possible strategies going forward for online marketing, so from our point of view the selected experts do have exactly the knowledge and experience that we are looking for.

Another problem could possibly arise from the selection of the experts on the panel, according to Robert Kaiser.\footnote{See \textit{Kaiser, R. (2014)}: Qualitative Experteninterviews, pp. 132-136 \cite{expertInterviews}} By using a rather diverse panel and focus on practical experience in the field, we hope to avoid this. Our panelists cover all relevant technical and legal aspects, but do not belong to the same opinion bubble, so that we can be relatively sure that we're avoiding too strong a bias here. We're also conducting the interviews one by one, to avoid that our experts influence each other.

The final problem that we want to address, is a possible lack of theory in the analysis and a too high an emphasis on the interview itself.\footnote{See \textit{Kaiser, R. (2014)}: Qualitative Experteninterviews, pp. 144-146 \cite{expertInterviews}} Focusing on possible strategies to deal with the demise of the tracking cookie has indeed potential to stray from the wider field of knowledge available. We have laid a solid foundation in the previous chapter on the basics of online marketing and covered the majority of tools at hand. We expect the outcome of the analysis will be within the parameters set by the research question and prove to be relevant for it. As far as our research question is concerned, we are looking for very practicable and actionable strategies for the future, and our panel seems to have the right knowledge for this.

\subsection{Expert Panel}

To achieve all of this and get a balanced view and a broad range of expertise, we have selected three well-knows experts in their field, from Cologne and the surrounding areas. All three of them are highly regarded experts in their field and fully qualified to offer insight and guidance on the subject.

Let's have a look at each of them more closely:

\begin{itemize}
 \item \href{https://www.linkedin.com/in/renate-schmid-535233113/}{Renate Schmid}
 \item \href{https://www.linkedin.com/in/roman-pusep-36b33374/}{Roman Pusep}
 \item \href{https://www.linkedin.com/in/eicker/}{Gerrit Eicker}
\end{itemize}

Renate is a lawyer at \href{https://www.wbs-law.de/}{Wilde Beuger Solmecke}, a renowned and leading law firm in the field of media and copyright law, located in Cologne. With \href{https://www.youtube.com/user/KanzleiWBS}{Kanzlei WBS} they operate the biggest German-language law channel on YouTube and are widely recognized as experts in their field. Throughout the last couple of years, WBS has actively participated in the discussions about copyright law and Article 17 (the article formerly known as 13)\footnote{See \textit{Solmecke, C. (2021)}: Die Uploadfilter sind da \cite{article17}} and are known as thought-leaders in the field of e-privacy. Other fields of expertise in addition to media and copyright law, according to their website, are IT and internet law, data protection and security regulations, e-commerce and competition law. Renate's personal fields of expertise are media law, data security and protection, and mediation; they hold a TÜV certification as data protection expert and are fluent in French and English.

Roman is a partner in the law firm \href{https://www.werner-ri.de/}{Werner Rechtsanwälte Informatiker}. Werner RI is a law firm providing legal advice on commercial law, with a strong focus on IT, data protection and internet law; they are widely known as thought-leaders on law in the field of digitization and frequently offer seminars and workshops on the intersection of IT and law. Romans personal fields of expertise are IT law, commercial and corporate law, and they also hold a TÜV certification as data protection expert; they are fluent in Russian.

Gerrit is the owner of the agency \href{https://eicker.digital/}{eicker.digital - Wir sprechen Online}. The agency is focused on digital communication and is known as a thought-leader in the field of digitization and conversation as well as knowledge management. With \href{https://www.youtube.com/eickertv}{eicker.TV} they run a well-known German-language channel for up-to-date information on digital marketing and communication and regularly publish on the major marketing networks. Gerrit is heavily invested in web analytics and process optimization. They coined the term nethnology, or the ethnology on the net, encompassing the social, cultural and economic life of netizens. The main method of nethnology, according to their web page, is participant observation on the web and social networks, making them an ideal expert candidate for online marketing and especially analytics.

Between the tree of them we are fortunate to have covered a broad range of expertise on the fields that we need to look into for the future of personalized or targeted advertising. Our experts cover a broad area, from data protection laws and civic society all the way down to digital communication. 

This diverse panel will guarantee a well-rounded outcome and meaningful results.

\subsection{Interview Questionnaire}

After selecting our panel, the next big part in setting up the interview was to create a questionnaire.

The interview questions are supposed to act as prompts and guides for the experts, they are not meant as a survey; the aim of the questions is to have the interviewees tell their own stories, in their own terms, and share their own expertise.\footnote{See \textit{Nelson, T. (2012)}: Some Strategies for Developing Interview Guides \cite{qualiStrategies}}

With the technology still being in flux and the legal framework still developing, we want to focus our panel questions on the the one aspect that we believe will not likely change in the near future, and that is the three different types of data that we have outlined above.

Let's quickly rehash the various data types: First party data is data directly collected from the users on your page, second party data is someone else's first party data and third party data is aggregated date you buy from a data exchange or management platform.\footnote{See \textit{Lotame (2021)}: 1st Party Data, 2nd Party Data, 3rd Party Data: What Does It All Mean? \cite{lotameRehash}}

Regardless of future development, we can safely assume that this classification will hold for a couple of years and will be a good basis for further analysis. The three types of data are independent of any technology; they are also treated differently in current legislation.

It's important to keep in mind that type of cookies that is at the heart of this paper, the third party tracking cookie, is only relevant for third party data.

For the interview questions, we'll stick to using direct questions only, and keep them simple and straightforward. We want to gather as much information from our panelists as possible, and let them answer in their own words and from their own experience, without leading them towards a certain outcome or opinion.\footnote{See \textit{Kaiser, R. (2014)}: Qualitative Experteninterviews, p. 68 \cite{expertInterviews}} 

It's the unfiltered narrative of our experts that we are interested in.

Firstly, we start with asking about first party data. We want to gather information on how our experts view the field of first party data in the current environment and in the near future. We expect the emphasis in their answers on the legal and technical aspects, mainly around GDPR/DS-GVO and analytics.

\begin{itemize}
 \item In the current environment, from your point of view, what's the primary legal basis for collecting first-party data?
 \item In the near future (one to three years), how do you see the development of the legal situation for collecting first-party data?
\end{itemize}
 
From the answers we expect additional information on the current legal frameworks, data privacy aspects, and a commercial view from a marketing viewpoint. Also, we expect an outlook on the possible developments in the next couple of years.
 
Secondly, we ask the same question again, this time for second party data:

\begin{itemize} 
 \item In the current environment, from your point of view, what's the primary legal basis for obtaining second-party data?
 \item In the near future (one to three years), how do you see the development of the legal situation for obtaining second-party data?
\end{itemize}

Here, we are expecting additional information especially on the possible roles of trustees and data processors, in addition on more general views on second party data itself and on how it could possibly play a bigger role in online marketing in the future. We're also hoping to gather information on the current use of second party data in online marketing and advertising right now.
 
Thirdly, we ask the final set of questions on third party data:

\begin{itemize} 
 \item In the current environment, from your point of view, what's the primary legal basis for obtaining third-party data?
 \item In the near future (one to three years), how do you see the development of the legal situation for obtaining third-party data?
\end{itemize} 
 
The use of third party data is the most contentious usage, here we are looking for more insights especially into the privacy aspects, together with more general views on the legality of the current approaches and possible options for the future.

We also hope to get some balanced information on the possible sources for third party data, and the origins of the third party data, one of which is the tracking cookie, whose demise is at the root of this paper.

To finish, we'll ask the experts to speculate about the future of targeting and personalized advertising in online marketing:
 
\begin{itemize} 
 \item Taking a wild guess, from a legal point of view, where do you think targeting on the web will be in ten years?
\end{itemize}

We do not have any firm expectations on the answers here, we just want to learn what our experts think about the future of online marketing and where they see the development going. We also hope to gain some insight on the future of personalized advertising and the underlying data basis and algorithms.

\subsection{Interview Execution}

To finish, here is some technical information about the expert interviews:

The interview with Renate Schmid was conducted in English and entirely through Email; a summary write-up will be included with the thesis data.

The interviews with Roman Pusep and Gerrit Eicker were both conducted each over a 90 minutes video call in German. The call was made and recorded with \href{https://zoom.us/}{Zoom} and then manually transcribed using \href{https://f4x.audiotranskription.de/}{f4x Spracherkennung} as a starting point; the transcript will be included with the thesis data.

Where necessary for the analysis, translation from German to English was performed with support from \href{https://www.deepl.com/en/translator}{DeepL} and \href{https://translate.google.de/}{Google Translate}.

After transcribing the expert interviews, we will now analyze the outcomes in the next chapter and then come up with an answer to our question about possible future strategies for online personalized advertising in the final chapter.
