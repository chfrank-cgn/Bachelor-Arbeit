%
%	Resultate
%

\pagebreak
\section{Resulting Strategies}

\onehalfspacing

\subsection{Use Your Own Data}

All our panelists agree that working with your own, first party daty, is the best option going forward. This is especially true for \textbf{Gerrit}, who sees working with first party data as a must for any serious marketeer in the online business. Or, as \href{https://tealium.com/}{Tealium} so succintly put it, "In a World Without Third-Party Cookies, a First-Party Data Strategy Takes the Cake".\footnote{\textit{Tealium (2021)}: First Party Data Strategy Takes the Cake \cite{firstCake}}

To work with our own data, we first need a data strategy.\footnote{See \textit{Dixon, M. (2020)}: How to build a data strategy \cite{dataStrategy}}

The first step will always be to identify the business need, in this case for our online marketing campaigns. We need to identify what we want to achieve, for example increase sales or diversify our customer base, just to name a few. Once we have these goals defined, we need to identify all the sources of data that we already have, and possibly add new ones through new tools for our web site or shop.

We do not want to go into too much detail on defining a data strategy here, there are however, tow points that are quite pertinent for the field of online marketing.

One is quite obvious, but needs special consideration: To use data to improve our customer experience, we need to be able to identify our customer. Mere analytics will not be enough - even though analytics will tell us which part of our site performed well and why, it will not tell us enough about the individual users.

Once we start identifying our customers, though, for example through a login or through placing a non-functional first party cookie, we enter the realm of processing personally identifiable data. Handling PII in Europe is governed by GDPR/DS-GVO and additionaly through the BDSG in Germany. This is a complex topic in itself, and completely out of scope for this paper.

To process PII, we need two things, among others: We need to ask for consent and we need to have a data protection declaration.

Obtaining informed consent, as required by law, can be difficult and \textbf{Roman} has a very distinct view on how to obtain such consent. Similarly, a data protection declaration can also be quite difficult to craft; here's an \href{https://blueropeconsultonline.de/datenschutz/}{example} from \href{https://blueropeconsultonline.de/}{Bluerope Consult} that was created with support from WBS Law, who were also on the expert panel on this paper.

Once we have all this in place and can identify our customers, we can now plan our advertising and marketing strategies much better and tailor them precisely for each of our customers - we're delivering personalized advertising, without the use of third party data and completly within the boundary of the law.

There are many also other fun and creative things we can do with our own data, such as following up on a sale or a series:

\begin{figure}[H]
\centering
\caption {Netflix Continue Watching Seraph of the End}
\includegraphics[scale=0.6]{images/continue-seraph.png}
\label{fig:seraph}
\end{figure}

The other point that we need to make is that with the start of processing PII, the level of data protection and data security that we need to provide in IT immediately increases a lot. Processing PII will open up a whole new avenue of attack vectors and increase the number of potential threads, as outlined in NIST's Guide to Data-Centric System Threat Modeling.\footnote{See \textit{Souppaya, M. (2016)}: Draft NIST Special Publication 800-154 \cite{sp800_154_draft}}

Through rigorous scientific analysis we have now identified a most promising strategy for online marketing in a cookie-less era. During our analysis, a number of other options emerged that we will now present, in no particular order.

\subsection{Cooperate on Data}

Text

\subsection{Use Data Clean Rooms}

Text

Dylan Siriwardana makes a passionate plea to consider data clean rooms as strategic choice after for the post-cookie era.\footnote{See \textit{Siriwardana, D. (2020)}: Data Clean Rooms als nützliches Werkzeug in der Post-Cookie-Ära \cite{dataClean}}

\subsection{Cooperate Using Contract Processor Agreements}

Text

\subsection{Stick to Contextual Advertising Only}

Text

\subsection{Continue with Third Party User Profiles}

Text

As we can see in a recent article on \href{https://www.vice.com/en}{Vice}, there seem to be many more types data on users available on the web than we might think.\footnote{See \textit{Cox, J. (2021)}: Google Bans Location Data Firm \cite{dataClean}}
