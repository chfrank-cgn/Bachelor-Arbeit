%
%	Fazit
%

\pagebreak
\section{Summary}

\onehalfspacing

\subsection{Targeting}

With the demise of the ubiquitous tracking cookie, some other form of ID will evolve to support targeted advertising. It is crucial for the future of the advertising business to have some form of ability to show meaningful ads - and it's ads that power the web.

Google FLoC is facing a lot of push back, so we will most likely see a different method evolve over the next couple of months. It will be interesting to see how the various approaches evolve and how in the end we will see ads that matter been delivered to users.

Even though none of the panelists were focusing on contextual advertising, I still see this as a very promising approach in delivering meaningful ads to users while full respecting their privacy.

\subsection{Method}

Given the volatile state of the subject in 2021, using a qualitative approach has proven very beneficial. Deviating from the classical approach of categorization and employing a summative technique have also greatly contributed to the overall result and enabled us to work with a relatively small set of experts.

The panel's responses were quite enlightening as they came from very diverse backgrounds, in summary they shared a good overview on the current situation and the possible paths forward. 

\subsection{Outlook}

We will see in 2022 or 2023 which direction advertising will take, we can be sure though that it will remain an integral and important part of the world-wide web infrastructure.

A sentiment that all panelists shared was the huge importance of first party data and it's increasing role in the advertising industry - a key takeaway for the future for all marketeers is to really become familiar with the provisions of the GDPR and BDSG.

Happy Advertising!
