%
%	Fazit
%

\pagebreak
\section{Summary}

\onehalfspacing

\subsection{Targeting}

With the demise of the ubiquitous tracking cookie, some other form of ID will evolve to support targeted advertising. It is crucial for the future of the advertising business to have some form of ability to show meaningful ads - and it's ads that power the web.

Google FLoC is facing a lot of push back, so we will most likely see a different method evolve over the next couple of months. It will be interesting to see how the various approaches evolve and how in the end we will see ads that matter been delivered to users.

There's a strong push to achieve a balance between the privacy needs of the uses and the targeting needs of the advertisers. It is quite possible that a new technology emerges that we have not seen so far - with the current technologies and the current legal framework, striking a compromise will be a difficult task and something that will keep marketeers busy for quite some time to come.

From the answers of our panelists, we can see that relying on first party data is most likely a very promising strategy, as it fits well into the current and emerging legal frameworks. Possibly a institution will step up and regulate the use of second party data in clean rooms, but there's no indication yet that this will happen soon. On third party data we can safely assume that the days of unfettered use are gone, or will be gone soon. 

In addition, even though none of the panelists were focusing on contextual advertising, I still see contectual advertising as a very promising approach in delivering meaningful ads to users, because you can do so while fully respecting the users' privacy.

In summary, we can conclude that working with your own data might be the best possible approach going forward.

\subsection{Method}

Given the volatile state of the subject in 2021, using a qualitative approach has proven very beneficial. Deviating from the classical approach of categorization and employing a summative technique have also greatly contributed to the overall result and enabled us to work with a relatively small set of experts.

The experts selection across a broad and divers range of backgrounds also helped achieving a well rounded result; in summary they shared a good overview on the current situation and the possible paths forward. 

\subsection{Outlook}

We will see in 2022 or 2023 which direction targeted advertising will take, we can be sure though that it will remain an integral and important part of the world-wide web infrastructure.

A sentiment that all panelists shared was the huge importance of first party data and it's increasing role in the advertising industry - and a key takeaway for the future for all German marketeers is to really become familiar with the provisions of the GDPR and BDSG.

Happy Advertising!
