%
%	Praxisbezug
%

\pagebreak
\section{Interview Analysis}

\onehalfspacing

\subsection{Regarding First Party Data}

\subsubsection{Paraphrased Interview Content}

\textbf{Renate} points out that companies should be aware of Art 6 I lit f) DS-GVO as the important legal basis for all technical cookies, and Art 6 I lit a) DS-GVO for any cookie that does not have a technical function, but is used for other purposes, for example user behavior.

For the future they see a strong focus on first party cookies and data, because people will prefer privacy.

\textbf{Roman} also puts a strong emphasis on the DS-GVO. Key factor from their point of view is the question, whether one is dealing with personally identifiable information (PII) or not, as this would decide whether the DS-GVO is applicable at all. Based on a recent decision of the EuGH, all data that includes an IP address would be considered personally identifiable - on the internet, that unfortunately applies to a lot of data items, possibly including cookies. Once we've clearly established that the DS-GVO applies to the data in question, the next step would be to determine whether storing and processing it is permissible withing the boundaries of the law. As of today, the primary aspect to consider here seems to be informed consent.

Personally identifiable does mean that it can be traced back to a natural person - this does not necessarily apply to the advertising identifier that we have looked at before, such as FLoC or IDFA.

\textbf{Gerrit} sees first party data as the most important asset for any company. As they collect and own the data, it's part of their core business. Processing first party data from a legal and privacy aspect is relatively easy, it just requires transparency and informed consent. The upcoming TT-DSG will allow functional cookies - for anything else, you'll still need informed consent. 

For any form of marketing, you'll need data; the best data is the data you own. Unlike with second or third party data, once you have consent you can use the data of your visitors or customers to provide the best personalized experience. A good example is Amazon, that only relies on first party data in their advertising.

\subsubsection{Generalization}

All panelists agree on the fact, that first person data is by far the most important data for a business. With GDPR (and similar legislation across the world) we have a solid framework to process data that we own, and once we obtain consent we can work unencumbered with the data, as long as we don't share it.

Using the analytics tools that we previously discussed, we take ownership of our web site or web shop, and start marketing campaigns tailored to our customers, based on the data we own. 

By keeping our data to ourselves, we can maintain the necessary and desired level of privacy for our users.

\subsection{Regarding Second Party Data}

\subsubsection{Paraphrased Interview Content}

\textbf{Renate} sees Art 6 I lit a) DS-GVO as an issue for companies, because users do not have to expect that their data is transferred to a second party; they see no change on this issue in the near future.

\textbf{Roman} also sees an issue with the DS-GVO here, split between the party that sends the data and the party that receives the data. For the sender, it would be necessary to obtain an explicit consent of every person affected - a blanket consent for sharing for marketing purposes (as an example) won't suffice. The same for the receiver - to process the data, it would need consent of all parties involved, which will be very difficult if not impossible to get.

One way around this could be a contract processor agreement, in which case the second party would become an extension of the first party and there would be no data transfer, from a legal point of view.

\textbf{Gerrit} sees the biggest problem in data handling once you start combining data or involving third parties. A possible solution could be intermediate data processors or clean rooms, but they do not see a viable business model for such a service yet.

Once data from different companies gets combined, the danger is that super profiles could emerge, that allow for easy identification of individual persons.

Having said this, in an ideal world from their point of view, all companies would have solid first party data and there were services available that would allow the companies to combine their data in a secure and privacy-aware way. 

\subsubsection{Generalization}

Sharing personally identifiable information with another company is a big privacy concern and runs afoul of most data privacy laws. Without established intermediaries or data clean rooms, working with second party data would require consent of every individual involved, which might be quite difficult to get.

If there was a solid legal framework for such intermediary services, it could in the future turn into a good option for online marketing.

\subsection{Regarding Third Party Data}

\subsubsection{Paraphrased Interview Content}

\textbf{Renate} again sees an issue with Art 6 I lit a) DS-GVO here, as it might be quite difficult to get consent from all parties. They expect it to become even more difficult in the future, because of potentially strict rules of the upcoming e-Privacy act.

\textbf{Roman} has a slightly different view and questions whether the mere use of someone else data could be a violation of the DS-GVO. If a company were to ask Google Ads, for example, to place an ad for a car to users that match the car aficionados criteria, they would not be processing personally identifiable data and hence it would not an issue under the DS-GVO. Google showing an ad to their users on their search results page is, by itself, not an issue in regards to data privacy, neither for Google nor for the company placing said advertisement.

If both parties desire a closer cooperation, a contract processor agreement and a solid mechanism to obtain consent could work.

\textbf{Gerrit} sees working with third party data as the biggest issue of all, especially for online marketing. Obtaining public third party data is easy, and some data is even made available voluntarily through the use social media for example.

Given the vast processing power of today's compute infrastructure, companies can combine individual profiles on almost all individual users - sometimes the profiles might even be more accurate than what the users know about themselves.

What's missing from their point of view is comprehensive regulation on the use of big data on personal profiles, but they do not see a possibility to stop the usage of the data anymore; they even argue that there's no need for further data collection to continue to feed and grow the underlying algorithms that govern our lives.\footnote{See \textit{Kling, M.-U. (2019)}: QualityLand Band 1 \cite{qualityLand}}

\subsubsection{Generalization}

Dealing with third party data has two aspects: One is the enormous amount of personally identifiable information that's already available online, through social media and through Data Management Platforms - processing this data is pretty much unregulated and it does not undergo any quality check.

The other aspect is that we might be using such data when placing marketing campaigns through a broker, such as Google Ads or Facebook Ads (which we covered previously) - even though it might no be a privacy problem to use a broker's profiles, we have no upfront control over the quality of the data, unlike if we were to use our own, first party data.

\subsection{On a 10-year Outlook}

\subsubsection{Paraphrased Interview Content}

\textbf{Renate} expects targeting to become more subtle and more hidden, trying to circumvent privacy regulations.

\textbf{Roman} expects targeting to become easier and foresees a simple mechanism to obtain consent, possibly through a general or per web page browser setting. From their point of view, as a pragmatist, the market demand for targeted advertising will most likely shape legislation in its favor.

\textbf{Gerrit} see the biggest task for the future to educate the people on the use of modern media ("Medienkompetenz").

Furthermore they do not believe that targeting in its current form will be necessary in the long run. With the rise of the ubiquitous digital assistants that can anticipate the needs of the individual user, targeting will become much less important and the systems will be able to use its own feedback to improve and learn.

Nevertheless, given the speed at which technology develops, they expect targeting and tracking to further grow and become all encompassing.

\subsubsection{Generalization}

Tracking and profiling seems to be here to stay. It might become more hidden, or handed over to big data algorithms, but it seems to be too late to stop the proliferation of online user profiles. 

Targeted advertising will thus remain, it might change with the proliferation of digital assistants and become less of a manual process of campaign placement but more of an algorithmic function in the shopping metaverse.

\subsection{Conclusion}

Summary

\pagebreak
\section{Results}

\onehalfspacing

\subsection{Strategy: Use your own data}

Text

\subsection{Strategy: Cooperate on data through data clean rooms}

Text

\subsection{Strategy: Cooperate on data through contract processor agreements}

Text

\subsection{Strategy: Stick to contextual advertising only}

Text

\subsection{Strategy: Continue with user profiles available for sale}

Text

