%
%	Praxisbezug
%

\pagebreak
\section{Interview Analysis}

\onehalfspacing

\subsection{Regarding First Party Data}

\subsubsection{Paraphrased Interview Content}

\textbf{Renate} points out that companies should be aware of Art 6 I lit f) DS-GVO as the important legal basis for all technical cookies, and Art 6 I lit a) DS-GVO for any cookie that does not have a technical function but is used for other purposes, for example, user behavior.

They see a strong focus on first-party cookies and data for the future because people will prefer privacy.

\textbf{Roman} also puts a strong emphasis on the DS-GVO. The key factor from their point of view is the question, whether one is dealing with personally identifiable information (PII) or not, as this would decide whether the DS-GVO is applicable at all. 

Based on a recent decision of the EuGH, all data that includes an IP address would be considered personally identifiable - on the internet, which unfortunately applies to a lot of data items, possibly including cookies. Once we've clearly established that the DS-GVO applies to the data in question, the next step would be to determine whether storing and processing it is permissible within the boundaries of the law. As of today, the primary aspect considered here by most parties to judge the legality seems to be informed consent, although the law offers other alternatives, such as justified interest.

Personally identifiable data does mean that the data can be traced back to a natural person - this does not necessarily apply to the advertising identifier that we have looked at before, such as FLoC or IDFA. If they are truly anonymous, such identifiers would hence not be considered personally identifiable data and not fall under the provisions of the DS-GVO.

\textbf{Gerrit} sees first party data as the most important asset for any company. As they collect and own the data, it's part of their core business. Processing first party data from a legal and privacy aspect is relatively easy; it just requires transparency and informed consent. The upcoming TT-DSG will allow functional cookies; for anything else, you'll still need informed consent. 

All data protection and privacy legislation are focused on processing personally identifiable data from a company's own source, and thus the process of dealing with such regulations is well established.

For any form of marketing, you'll need data; according to Gerrit, the best data is the data you already own. Unlike with second or third party data, once you have consent, you can use the data of your visitors or customers to provide the best personalized experience.  good example is Amazon, which only relies on first party data in its advertising and is hugely successful.

\subsubsection{Generalization}

All panelists agree on the fact that first person data is by far the most essential data for a business. With GDPR/DS-GVO (and similar legislation across the world), we have a solid framework to process data that we own. Once we obtain consent, we can work unencumbered with the data, as long as we don't share it and stick to the communicated use.

Using the analytics tools that we previously discussed, we can take ownership of our web site or web shop and analyze its performance. We can create and run marketing and advertising campaigns tailored to our customers, based on our own data and entirely within the boundaries of the relevant data protection laws.

By only processing our data by ourselves or through approved contract processors, we can maintain the necessary and desired level of privacy for our users.

\subsection{Regarding Second Party Data}

\subsubsection{Paraphrased Interview Content}

\textbf{Renate} sees Art 6 I lit a) DS-GVO as an issue for companies because users do not have to expect that their data is transferred to a second party; they see no change on this issue in the near future.

\textbf{Roman} also sees an issue with the DS-GVO here, split between the party that sends the data and the party that receives the data. For the party that sends the data, it would be necessary to obtain the explicit consent of every person affected - a blanket consent for sharing for marketing purposes (as an example) won't suffice. The same for the party that receives the data - to process the data, it would need the consent of all parties involved, which will be very difficult if not impossible to get.

Around consent and consent forms, Roman points out that there is a common practice of enticing users to agree without fully understanding the choices - from his point of view, this is something that lacks transparency and would invalidate the consent.

One way around this could be a contract processor agreement, in which case the second party would become an extension of the first party, and there would be no data transfer, from a legal point of view, but Roman does not see this as a feasible option in the real world.

\textbf{Gerrit} too sees the biggest problem in the data handling once you start combining data or involving third parties. A possible solution could be intermediate data processors or clean rooms, but they do not see a viable business model for such a service yet.

Once data from different companies get combined, according to Gerrit, the danger is that super profiles could emerge, which would allow for easy identification of individual persons and would run certainly cause issues with the existing set of data protection regulations.

Having said this, in an ideal world, from their point of view, all companies would have solid first party data, and there were services available that would allow the companies to combine their data in a secure and privacy-aware way. 

\subsubsection{Generalization}

Sharing personally identifiable information with another company is a big privacy concern and runs afoul of most data privacy laws. Without established intermediaries or data clean rooms, working with second party data would require the consent of every individual involved, which might be pretty challenging to get.

If there was a solid legal framework for such intermediary services, it could in the future turn into a good option for online marketing.

\subsection{Regarding Third Party Data}

\subsubsection{Paraphrased Interview Content}

\textbf{Renate} again sees an issue with Art 6 I lit a) DS-GVO here, as it might be quite difficult to get consent from all parties. They expect it to become even more difficult in the future because of potentially strict rules of the upcoming e-Privacy act.

\textbf{Roman} has a slightly different view and questions whether the mere use of someone else data could be a violation of the DS-GVO. If a company were to ask Google Ads, for example, to place an ad for a car to users that match the car aficionados criteria, they would not be processing personally identifiable data. Hence it would not become an issue under the DS-GVO. Google showing an ad to their users on their search results page is, by itself, not an issue in regards to data privacy, neither for Google nor for the company placing the said advertisement.

If both parties desire closer cooperation, a contract processor agreement and a solid mechanism to obtain consent could work but create its own legal conundrums. A contract processor agreement shifts the responsibility for all actions to the originator of the contract, and also all liability. According to Roman, this makes such a scenario challenging to set up and rather unlikely.

\textbf{Gerrit} sees working with third party data as the biggest issue of all, especially for online marketing. Obtaining public third party data is easy, and some data is even made available voluntarily through the use of social media, for example.

Given the vast processing power of today's computing infrastructure, companies can combine individual profiles on almost all individual users. Sometimes, the profiles might even be more accurate than what the users know about themselves. From Gerrit's point of view, a lot of the users on the world wide web do not know how much data they have already shared and how much money this data is worth to the companies creating the profiles.

What's missing from their point of view is comprehensive regulation on the use of big data on personal profiles, but they do not see a possibility to stop the usage of the data anymore; they even argue that there's no need for further data collection to continue to feed and grow the underlying algorithms that govern our lives.\footnote{See \textit{Kling, M.-U. (2019)}: QualityLand Band 1 \cite{qualityLand}} Predictive analysis can enable companies to gain a much better picture of us than what we have ourselves.

According to Gerrit, the deed is done, and there's no way to restrict or curb the usage of third party data in the future.

\subsubsection{Generalization}

Dealing with third party data has two aspects: One is the enormous amount of personally identifiable information that's already available online, through social media, and through Data Management Platforms - processing this data is pretty much unregulated. 

It also does not undergo any quality check, so the usage is not without risk.

The other aspect is that we might be using such data when placing marketing campaigns through a broker, such as Google Ads or Facebook Ads (which we covered previously) - even though it might no be a privacy problem to use a broker's profiles, we have no upfront control over the quality of the data, unlike if we were to use our own, first party data.

Without tracking cookies, and thus without an easy-to-access browsing history, there's a possibility that the profiles will become less accurate in the future; however, advances in big data and machine learning might offset that loss and lead to even more detailed user profiles in the future.

\subsection{On a 10-year Outlook}

\subsubsection{Paraphrased Interview Content}

\textbf{Renate} expects targeting to become more subtle and more hidden, trying to circumvent privacy regulations.

\textbf{Roman} expects targeting to become easier and foresees a much easier and simple mechanism to obtain consent, possibly through a general or per web page browser setting.

From their point of view, as a pragmatist, the market demand for targeted advertising will most likely shape legislation in its favor; they expect the market to instead grow than shrink.

\textbf{Gerrit} see the most significant task for the future to educate the people on the use of modern media ("Medienkompetenz").

Furthermore, they do not believe that targeting in its current form will be necessary for the long run. With the rise of the ubiquitous digital assistants that can anticipate the needs of the individual user, targeting will become much less important, and the systems will be able to use their own feedback loops to improve and learn.

Nevertheless, given the speed at which technology develops, they expect targeting and tracking to grow further and become all-encompassing. They expect that laws and regulations will have to play catch up.

\subsubsection{Generalization}

Tracking and profiling seem to be here to stay. It might become more hidden or handed over to big data algorithms, but it seems to be too late to stop the proliferation of online user-profiles and their use for personalized advertising.

Targeted advertising will thus remain; it might change with the proliferation of digital assistants and become less of a manual process of campaign placement but more of an algorithmic function in the shopping metaverse.

We can expect the legislation to lag behind the technical capabilities.

\subsection{Conclusion}

All our panelists agree that working with first party data is the most promising way forward - working with first party data will feature prominently in our resulting strategies. Having said this, dealing with first party data also requires the most extensive effort of all available options. 

To gather first party data on the web, the web site needs to be correctly instrumented for analytics, and KPIs and measurements need to be defined and set up for all marketing and sales campaigns. As it is first party data, i.e., data we own, no other party can collect this data but ourselves.

From a legal point of view, working with first party data is easy and well defined.

The concept of data clean rooms to deal with second party data seems to be very interesting, but for the time being lacks both a solid legal framework as well as a solid commercial offering, according to our panelists.

Contract processing agreements are another option to get around the legal issues involved with sharing data. It does, however, not seem to be a feasible option in practical terms, as it carries a significant liability risk and costly contract negotiations and setup.

All panelists also agree on the fact that the proliferation of readily available user profiles on the web can no longer be stopped and that the upcoming demise of the tracking cookie might be offset by an increase in computing power and better machine learning algorithms. They see the possibilities that augmented user profiles might actually become more detailed than they are right now.

One other point of consideration could be the recent publication of Assessment Report \#6 by the Intergovernmental Panel on Climate Change, which might affect our outlook on the future of online marketing, but we will leave that point open for now and concentrate on our resulting strategies instead.\footnote{See \textit{von Juterczenka, J. (2021)}: Was im sechsten Weltklimarat-Sachstandsbericht steht \cite{ipccAr6}}

The next and final chapter will offer a couple of strategies for personalized advertising and online marketing without relying on data gathered from tracking cookies. These designs are based on the recommendations from our experts above.
