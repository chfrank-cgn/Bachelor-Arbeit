%
%	Praxisbezug
%

\pagebreak
\section{Interview Analysis}

\onehalfspacing

\subsection{Regarding First Party Data}

\subsubsection{Paraphrased Interview Content}

\textbf{Renate} points out that companies should be aware of Art 6 I lit f) DS-GVO as the important legal basis for all technical cookies, and Art 6 I lit a) DS-GVO for any cookie that does not have a technical function, but is used for other purposes, for example user behavior.

For the future they see a strong focus on first party cookies and data, because people will prefer privacy.

\textbf{Roman} also puts a strong emphasis on the DS-GVO. Key factor from their point of view is the question, whether one is dealing with personally identifiable information (PII) or not, as this would decide whether the DS-GVO is applicable at all. Based on a recent decision of the EuGH, all data that includes an IP address would be considered personally identifiable - on the internet, that unfortunately applies to a lot of data items, possibly including cookies. Once we've clearly established that the DS-GVO applies to the data in question, the next step would be to determine whether storing and processing it is permissible withing the boundaries of the law. As of today, the primary aspect to consider here seems to be informed consent.

Personally identifiable does mean that it can be traced back to a natural person - this does not necessarily apply to the advertising identifier that we have looked at before, such as FLoC or IDFA.

\textbf{Gerrit} 

\subsubsection{Generalization}

First party data - the most important data

\subsection{Regarding Second Party Data}

\subsubsection{Paraphrased Interview Content}

\textbf{Renate} sees Art 6 I lit a) DS-GVO as an issue for companies, because users do not have to expect that their data is transferred to a second party; they see no change on this issue in the near future.

\textbf{Roman} also sees an issue with the DS-GVO here, split between the party that sends the data and the party that receives the data. For the sender, it would be necessary to obtain an explicit consent of every person affected - a blanket consent for sharing for marketing purposes (as an example) won't suffice. The same for the receiver - to process the data, it would need consent of all parties involved, which will be very difficult if not impossible to get.

One way around this could be a contract processor agreement, in which case the second party would become an extension of the first party and there would be no data transfer, from a legal point of view.

\textbf{Gerrit} 

\subsubsection{Generalization}

Could work with data brokers, data clean rooms or contract processors

\subsection{Regarding Third Party Data}

\subsubsection{Paraphrased Interview Content}

\textbf{Renate} again sees an issue with Art 6 I lit a) DS-GVO here, as it might be quite difficult to get consent from all parties. They expect it to become even more difficult in the future, because of potentially strict rules of the upcoming e-Privacy act.

\textbf{Roman} has a slightly different view and questions whether the mere use of someone else data could be a violation of the DS-GVO. If a company were to ask Google Ads, for example, to place an ad for a car to users that match the car aficionados criteria, they would not be processing personally identifiable data and hence it would not an issue under the DS-GVO. Google showing an ad to their users on their search results page is, by itself, not an issue in regards to data privacy, neither for Google nor for the company placing said advertisement.

If both parties desire a closer cooperation, a contract processor agreement and a solid mechanism to obtain consent could work.

\textbf{Gerrit} 

\subsubsection{Generalization}

Could work with contract processors, if there's a solid legal framework

\subsection{On a 10-year Outlook}

\subsubsection{Paraphrased Interview Content}

\textbf{Renate} expects targeting to become more subtle and more hidden, trying to circumvent privacy regulations.

\textbf{Roman} expects targeting to become easier and foresees a simple mechanism to obtain consent, possibly through a general or per web page browser setting. From their point of view, as a pragmatist, the market demand for targeted advertising will most likely shape legislation in its favor.

\textbf{Gerrit} 

\subsubsection{Generalization}

The future will be bright and DS-GVO interpreted in a way that allows for both privacy and advertising

\subsection{Overall Conclusion}

\begin{itemize}
 \item First party data 
 \item Data clean rooms
 \item Contract processors
 \item Contextual advertising
\end{itemize}
